\documentclass[../main.tex]{subfiles}
\begin{document}
\section{Deterministyczne automaty skończone - definicja, zastosowania.}

\textbf{Deterministyczny automat skończony} to abstrakcyjna maszyna o skończonej liczbie stanów, która
zaczynając w stanie początkowym czyta kolejne symbole pewnego słowa, po przeczytaniu każdego
zmieniając swój stan na stan będący wartością funkcji jednego przeczytanego symbolu oraz stanu
aktualnego. Jeśli po przeczytaniu całego słowa maszyna znajduje się w którymś ze stanów oznaczonych
jako końcowe, słowo należy do języka regularnego, do rozpoznawania którego jest zbudowana.

Deterministyczny automat skończony, podobnie jak inne automaty skończone może być reprezentowany za pomocą tabeli przejść pomiędzy stanami lub diagramu stanów. 

Deterministyczny automat skończony może być opisany przez $(A,Q,q_{0},F,d)$, gdzie:
\begin{itemize}
    \item $A$ jest skończonym alfabetem wejściowym
    \item $Q$ jest skończonym zbiorem stanów
    \item $q_{0}$ jest stanem początkowym, przy czym $q_{0} \in Q$
    \item $F$ jest zbiorem stanów akceptujących (końcowych), będącym podzbiorem $Q$.
    \item $d$ jest funkcją przejścia, przypisująca parze $(q,a)$ nowy stan $p$, w którym znajdzie się automat po przeczytaniu symbolu $a$ w stanie $q$.
\end{itemize}

\textbf{Słowa} mogą być dowolnymi symbolami, znakami należącymi do obranego alfabetu. Najprościej jest operować na słowach alfabetu binarnego.

\end{document}