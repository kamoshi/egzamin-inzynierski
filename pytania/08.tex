\documentclass[../main.tex]{subfiles}
\begin{document}
\section{Procesory typu RISC i CISC - charakterystyka, różnice.}

\textbf{CISC (Complex Instruction Set Computers)} - architektura którą cechuje duża liczba instrukcji,
występowanie złożonych i wyspecjalizowanych rozkazów, duża liczba trybów, mała optymalizacja
czasów wykonywania. Prosty kompilator, skomplikowane instrukcje. Można wykonywać operacje
bezpośrednio na pamięci.

\textbf{RISC (Reduced Instruction Set Computers)} - architektura o zredukowanej liczbie rozkazów, małej liczbie
instrukcji, które są mniej skomplikowane, złożony kompilator. Szybsze wykonywanie instrukcji.
Ograniczenie komunikacji procesora z pamięcią, przetwarzanie potokowe. Operacje na pamięci z
wykorzystaniem rejestrów.

\end{document}