\documentclass[../main.tex]{subfiles}
\begin{document}
\section{Programowanie obiektowe - podstawowe pojęcia, zastosowania.}

\textbf{Programowanie obiektowe} (OOP) to jeden z paradygmatów programowania, który opiera się o definiowanie programów przy pomocy obiektów - elementów łączących stan (dane, atrybuty) oraz zachowanie (procedury, metody). Podejście to różni się od programowania proceduralnego, gdzie dane oraz procedury nie są ściśle powiązane. W zamyśle programowanie obiektówe ma ułatwić pisanie, konserwację oraz wielokrotne uzycie programów lub ich fragmentów.

Największym atutem takiego podejścia do pisania programów komputerowych jest to, że ułatwia ono programowanie, analizę oraz modelowanie zgodnie z rzeczywistością, czyli tym jak ludzie odbierają świat. Przejawia się to tym, że obecnie programowanie obiektowe jest dominującym podejściem do wytwarzania oprogramowania.

W większości języków obiektowych występuje pojęcie klasy oraz obiektu.
\begin{itemize}
    \item \textbf{klasy} - definicje formatu danych oraz dostępnych procedur dla danego typu lub klasy obiektu, mogą same także zawierać dane oraz procedury (znane jako metody klas), na przykład, klasy zawierają dane oraz metody.
    \item \textbf{obiekty} - instancje konkretnych klas.
\end{itemize}

Niektóre języki nie umożliwiają tworzenia klas obiektów, wykorzystywane jest natomiast pojęcie prototypu. Przykładem języku który umożliwia korzystanie z paradygmatu obiektowego przy wykorzystaniu prototypów jest JavaScript.

Podstawowe założenia paradygmatu obiektowego:
\begin{itemize}
    \item \textbf{Abstrakcja} - Każdy obiekt w systemie służy jako model abstrakcyjnego „wykonawcy”, który może wykonywać pracę, opisywać i zmieniać swój stan oraz komunikować się z innymi obiektami w systemie bez ujawniania, w jaki sposób zaimplementowano dane cechy.
    \item \textbf{Hermetyzacja} - Czyli ukrywanie implementacji, enkapsulacja. Zapewnia, że obiekt nie może zmieniać stanu wewnętrznego innych obiektów w nieoczekiwany sposób. Tylko własne metody obiektu są uprawnione do zmiany jego stanu. Każdy typ obiektu prezentuje innym obiektom swój interfejs, który określa dopuszczalne metody współpracy.
    \item \textbf{Polimorfizm} - Referencje i kolekcje obiektów mogą dotyczyć obiektów różnego typu, a wywołanie metody dla referencji spowoduje zachowanie odpowiednie dla pełnego typu obiektu wywoływanego. Jeśli dzieje się to w czasie działania programu, to nazywa się to późnym wiązaniem lub wiązaniem dynamicznym.
    \item \textbf{Dziedziczenie} - Porządkuje i wspomaga polimorfizm i enkapsulację dzięki umożliwieniu definiowania i tworzenia specjalizowanych obiektów na podstawie bardziej ogólnych. Dla obiektów specjalizowanych nie trzeba redefiniować całej funkcjonalności, lecz tylko tę, której nie ma obiekt ogólniejszy. W typowym przypadku powstają grupy obiektów zwane klasami, oraz grupy klas zwane drzewami. Odzwierciedlają one wspólne cechy obiektów. 
\end{itemize}

\end{document}