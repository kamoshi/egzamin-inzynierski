\documentclass[../main.tex]{subfiles}
\begin{document}
\section{Przykładowe architektury komputerów: von Neumana, Princeton, Harvard.}

Architektura komputerowa definiuje to, w jaki sposób procesor jest połączony z pamięcią komputerową (RAM/ROM), celem wykonywania konkretnych programów komputerowych. Wyróżnić można dwie podstawowe architektury komputerowe: von Neumanna oraz Harvardzka.

\subsection{Architektura von Neumanna}
Charakteryzuje się tym, że:
\begin{itemize}
    \item komputer składa się z trzech części:
    \begin{itemize}
        \item Procesora - z częścią sterującą oraz arytmetyczno-logiczną
        \item Pamięci - przechowującej wszystkie dane oraz instrukcje wymagane do wykonania programu komputerowego
        \item Urządzenia wejścia / wyjścia
    \end{itemize}
    \item dane przechowywane są w tej samej formie, tak samo zakodowane - stąd zagrożenie wykonaniem danych jako instrukcji, lub też potraktowanie instrukcji jako danej.
    \item Pojedyncza szyna danych pomiędzy pamięcią a procesorem
\end{itemize}

\textbf{Zalety}
\begin{itemize}
    \item CU pozyskuje instrukcje i dane z jednej pamięci (bo można alokować dla instrukcji/danych wedle potrzeby)
    \item Organizacja pamięci według wizji programisty
\end{itemize}

\textbf{Wady}
\begin{itemize}
    \item Szeregowe przetwarzanie instrukcji
    \item Pojedyncza szyna
    \item Instrukcje dzieląc pamięć z danymi mogą zostać nadpisane
    \item Duży przesył danych pomiędzy pamięcią a CPU (brak cache)
\end{itemize}

\subsection{Architektura Harvardzka}
Charakteryzuje się tym, że:
\begin{itemize}
    \item komputer składa się z trzech części:
    \begin{itemize}
        \item Procesora - z częścią sterującą oraz arytmetyczno-logiczną
        \item Pamięci na dane
        \item Pamięci na intrukcje
        \item Urządzenia wejścia / wyjścia
    \end{itemize}
    \item instrukcje oraz dane przechowywane są w osobnych jednostkach pamięci komputerowej.
    \item każda pamięć jest obsługiwana osobno przez procesor więc możliwe jest jednoczesne pobieranie danych oraz instrukcji.
\end{itemize}

\textbf{Zalety}
\begin{itemize}
    \item Rozłączność intrukcji i danych = równoległy dostęp
    \item Różny rozmiar komórek pamięci
\end{itemize}

\textbf{Wady}
\begin{itemize}
    \item Wolna pamięć nie może być alokowana dla innego typu pamięci
    \item Droższa, potrzebne dwie szyny
\end{itemize}

\end{document}