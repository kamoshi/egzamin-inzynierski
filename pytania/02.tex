\documentclass[../main.tex]{subfiles}
\begin{document}
\section{Arytmetyka dwójkowa, funkcje boolowskie, tablice Karnaugh.}

\textbf{Arytmetyka dwójkowa}, także znana jako arytmetyka binarna, to arytmetyka opierająca się na dwójkowym systemie liczbowym. Znaczna większość systemów komputerowych do obliczeń wykorzystuje system dwójkowy (wyjątkiem są nieliczne komputery zaprojektowane zgodnie z logiką trójwartościową).

Zgodnie z tą arytmetyką wszystkie liczby przedstawiane są w postaci binarnej, czyli za pomocą zer oraz jedynek. A zatem wykorzystywany jest binarny system liczbowy, przy czym podstawą systemu binarnego jest liczba dwa (stąd nazwa system binarny).


\textbf{Funkcja boolowska} (inaczej funkcja logiczna) to dowolne odwzorowanie $f: X \rightarrow Y$, gdzie $B = \{0, 1\}$, $X$ jest podzbiorem $B^{n}$, zaś $Y$ jest podzbiorem $B^{m}$. Funkcja boolowska jest matematycznym modelem cyfrowego układu kombinacyjnego. Układy tego typu są wykorzystywane do konstrukcji między innymi multiplekserów, mikroprocesorów, do sterowania na przykład wyświetlaczami LED i w wielu innych urządzeniach elektronicznych.

Istnieje kilka odmiennych form zapisu funkcji boolowskich:
\begin{itemize}
    \item \textbf{Opis słowny} - może być stosowany w przypadku bardzo prostych funkcji boolowskich.
    \item \textbf{Tablica prawdy} - w tablicy zapisuje się wszystkie kombinacje zmiennych wejściowych oraz odpowiadające im wartości funkcji.
    \item \textbf{Mapa Karnougha} - jest to przekształcona tablica prawdy, przedstawiana w postaci prostokątnej tablicy, gdzie indeksy dwójkowe zostały pogrupowane tak, by spełniały właśności kodu Graya.
\end{itemize}

\textbf{Tablice Karnougha} wykorzystywane są do minimalizacji funkcji boolowskich. Na ogół znalezienie formuły minimalnej dla zadanej funkcji boolowskiej jest bardzo skomplikowanym problemem. Jednak jeśli funkcja ma małą liczbę zmiennych i zostanie zapisana w tablicy Karnougha, wówczas znalezienie minimalnej formuły odbywa się na drodze intuicyjnej. W przypadku większej liczby zmiennych logicznych wykorzystywana jest natomiast metoda Quinea-McCluskeya.

\end{document}